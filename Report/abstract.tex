\begin{abstract}
    \noindent 
    In this course we investigated the principle of a mass spectrometer using different samples. We first compared the two detectors that are built in. We then performed a couple of background measurements with different device settings in order to have them optimally selected in the measurements to come. Then we continued with measuring noble gases. The noble gases could be identified and, in the case of a gas mixture, the proportion of the components could be estimated. For the measured noble gases, the isotope ratio was also estimated and compared to literature values. In most cases the results agree with the literature and if not, possible reasons for the deviations could be identified. The decreasing amount of oxygen and increasing amount of carbon dioxide has also been measured in exhaled air. In addition, an air sample was examined whose carbon dioxide content was increased with the help of sparkling water. At last we determined the compounds of a deodorant and measured samples containing ethanol. For all measurements, it was determined by which substance combinations they can be explained. The literature values required for this were obtained from \texttt{NIST} \cite{NIST}.
\end{abstract}