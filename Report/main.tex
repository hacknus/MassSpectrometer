\documentclass[a4paper,12pt]{article}
\usepackage[utf8]{inputenc}

\usepackage{amsmath}
\usepackage{amssymb}
\usepackage{graphicx}
\usepackage{subfig}
\usepackage{cancel} % Formeln kürzen
\usepackage{xcolor}
\usepackage{braket}
\usepackage{float}
\usepackage{siunitx}
\usepackage{placeins}
\usepackage[T1]{fontenc} % für matlab
\usepackage[framed, numbered]{matlab-prettifier} % für matlab
\usepackage{circuitikz}

% read in csv tables
\usepackage{pgfplotstable,filecontents}
\pgfplotsset{compat=1.9}% supress warning
\usepackage{booktabs}
\pgfplotsset{
   /pgf/number format/textnumber/.style={
     fixed,
     fixed zerofill,
     precision=2,
     },
     /pgf/number format/textnumbertwo/.style={
     fixed,
     fixed zerofill,
     precision=4,
     },
}
\usepackage{datatool}



\usepackage[top=2.5cm, bottom=2.5cm, left=2.5cm, right=2.5cm]{geometry}
\usepackage{appendix}

\usepackage{biblatex}
\addbibresource{Literaturliste.bib}

\renewcommand*{\arraystretch}{1.3} % grössere Abstände in pmatrix

\usetikzlibrary{arrows,chains,matrix,positioning,scopes}


\tikzset{
 	>=triangle 45,
 	pos=.8,
 	photon/.style={decorate, thick,decoration={snake,segment length=5,post length=1mm,pre length = 1mm}},
	gluon/.style={decorate, thick, draw=black,
    		decoration={coil,amplitude=4pt, segment length=5pt,post length=1mm,pre length = 1mm}
	},
        particle/.style={ thick,
        postaction={decorate,
                    decoration={markings,mark=at position  0.5 with {\arrow[xshift=2pt +3\pgflinewidth]{<}}}
                   }
        },
        antiparticle/.style={ thick,
        postaction={decorate,
                    decoration={markings,mark=at position  0.5 with {\arrow[xshift=2pt +3\pgflinewidth]{>}}}
                   }
        }
}

\usetikzlibrary{patterns}
\usetikzlibrary{decorations.markings}

% defining the new dimensions and parameters
\newlength{\hatchspread}
\newlength{\hatchthickness}
\newlength{\hatchshift}
\newcommand{\hatchcolor}{}
% declaring the keys in tikz
\tikzset{hatchspread/.code={\setlength{\hatchspread}{#1}},
         hatchthickness/.code={\setlength{\hatchthickness}{#1}},
         hatchshift/.code={\setlength{\hatchshift}{#1}},% must be >= 0
         hatchcolor/.code={\renewcommand{\hatchcolor}{#1}}}
% setting the default values
\tikzset{hatchspread=3pt,
         hatchthickness=0.4pt,
         hatchshift=0pt,% must be >= 0
         hatchcolor=black}
% declaring the pattern
\pgfdeclarepatternformonly[\hatchspread,\hatchthickness,\hatchshift,\hatchcolor]% variables
   {custom north west lines}% name
   {\pgfqpoint{\dimexpr-2\hatchthickness}{\dimexpr-2\hatchthickness}}% lower left corner
   {\pgfqpoint{\dimexpr\hatchspread+2\hatchthickness}{\dimexpr\hatchspread+2\hatchthickness}}% upper right corner
   {\pgfqpoint{\dimexpr\hatchspread}{\dimexpr\hatchspread}}% tile size
   {% shape description
    \pgfsetlinewidth{\hatchthickness}
    \pgfpathmoveto{\pgfqpoint{0pt}{\dimexpr\hatchspread+\hatchshift}}
    \pgfpathlineto{\pgfqpoint{\dimexpr\hatchspread+0.15pt+\hatchshift}{-0.15pt}}
    \ifdim \hatchshift > 0pt
      \pgfpathmoveto{\pgfqpoint{0pt}{\hatchshift}}
      \pgfpathlineto{\pgfqpoint{\dimexpr0.15pt+\hatchshift}{-0.15pt}}
    \fi
    \pgfsetstrokecolor{\hatchcolor}
%    \pgfsetdash{{1pt}{1pt}}{0pt}% dashing cannot work correctly in all situation this way
    \pgfusepath{stroke}
   }
\pgfdeclarepatternformonly[\hatchspread,\hatchthickness,\hatchshift,\hatchcolor]% variables
   {custom south west lines}% name
   {\pgfqpoint{\dimexpr-2\hatchthickness}{\dimexpr-2\hatchthickness}}% lower left corner
   {\pgfqpoint{\dimexpr\hatchspread+2\hatchthickness}{\dimexpr\hatchspread+2\hatchthickness}}% upper right corner
   {\pgfqpoint{\dimexpr\hatchspread}{\dimexpr\hatchspread}}% tile size
   {% shape description
    \pgfsetlinewidth{\hatchthickness}
    \pgfpathmoveto{\pgfqpoint{0pt}{-0.15pt}}
    \pgfpathlineto{\pgfqpoint{\dimexpr\hatchspread+0.15pt+\hatchshift}{\dimexpr\hatchspread+\hatchshift}}
    \ifdim \hatchshift > 0pt
      \pgfpathmoveto{\pgfqpoint{0pt}{-0.15pt}}
      \pgfpathlineto{\pgfqpoint{\dimexpr0.15pt+\hatchshift}{\hatchshift}}
    \fi
    \pgfsetstrokecolor{\hatchcolor}
%    \pgfsetdash{{1pt}{1pt}}{0pt}% dashing cannot work correctly in all situation this way
    \pgfusepath{stroke}
   }

\usetikzlibrary{angles,quotes,arrows,decorations.pathmorphing,backgrounds,fit,positioning,shapes.symbols,chains}




% Default fixed font does not support bold face
\DeclareFixedFont{\ttb}{T1}{txtt}{bx}{n}{12} % for bold
\DeclareFixedFont{\ttm}{T1}{txtt}{m}{n}{12}  % for normal

% Custom colors
\usepackage{color}
\definecolor{deepblue}{rgb}{0,0,0.5}
\definecolor{deepred}{rgb}{0.6,0,0}
\definecolor{deepgreen}{rgb}{0,0.5,0}
\usepackage{hyperref}
\usepackage{listings}

% Python style for highlighting
\newcommand\pythonstyle{\lstset{
language=Python,
basicstyle=\ttm,
otherkeywords={self},             % Add keywords here
keywordstyle=\ttb\color{deepblue},
emph={MyClass,def,zeros,pass,sum,range,True,False},          % Custom highlighting
emphstyle=\ttb\color{deepred},    % Custom highlighting style
stringstyle=\color{deepgreen},
frame=tb,                         % Any extra options here
showstringspaces=false            % 
}}


% Python environment
\lstnewenvironment{python}[1][]
{
\pythonstyle
\lstset{#1}
}
{}



\begin{document}
    
    \pagenumbering{gobble} % keine Seitenzahl
    \begin{center}
    \vspace*{\fill} % zum vertikalen Zentrieren
    \LARGE{Lab Course II} \\
    \vspace{5mm}
    \Huge{Mass Spectrometer} \\
    \rule{10cm}{1pt} \\
    \vspace{2cm}
    \Large{Linus Stöckli \\Donat Hess} \\
    \vspace{1cm}
    \Large{Assistant: Nora Hänni} \\
    \vspace{1cm}
    \Large{University of Bern, \\ November 2020}
    \vspace*{\fill} % zum vertikalen Zentrieren
\end{center}
    \newpage
    %\null \newpage % eine leere Seite
    
    
    
    \begin{abstract}
    \noindent 
    In this course we investigated the principle of a mass spectrometer using different samples. We first compared the two detectors that are built in. We then performed a couple of background measurements with different device settings in order to have them optimally selected in the measurements to come. Then we continued with measuring noble gases. The noble gases could be identified and, in the case of a gas mixture, the proportion of the components could be estimated. For the measured noble gases, the isotope ratio was also estimated and compared to literature values. In most cases the results agree with the literature and if not, possible reasons for the deviations could be identified. The decreasing amount of oxygen and increasing amount of carbon dioxide has also been measured in exhaled air. In addition, an air sample was examined whose carbon dioxide content was increased with the help of sparkling water. At last we determined the compounds of a deodorant and measured samples containing ethanol. For all measurements, it was determined by which substance combinations they can be explained. The literature values required for this were obtained from \texttt{NIST} \cite{NIST}.
\end{abstract}
    \newpage
    
    \tableofcontents % Inhaltsverzeichnis
    \newpage
    
    \pagenumbering{arabic} % normale Seitenzahlen, beginnend mit Seite 1
    
    \section{Introduction} \label{sec:introduction}
Mass-spectrometers are widely used in different fields to determine the compounds of unknown substances and/or to measure the quantities thereof. 

% talk about different mass-spectrometers
% talk about different applications (stationary in a lab, or portable: army after a biochemical attack)
    \FloatBarrier
    
    \section{Theory} \label{sec:theory}
Mass-spectrometers are used to measure the amounts of different compounds in a gas. The gas is first ionized and then the different compounds are split up according to their charge-to-mass ratio. In this experiment we used a quadrupol-mass-spectrometer.

\subsection{Working principle}
    The gaseous probe is first ionized. This is done by electrons that are detached from a glowing filament and accelerated by an electric field. They then hit the compounds of the probe.  The newly generated ions are then also accelerated by the electric field and guided through a electrodynamic quadrupole field. The three components \eqref{eq:quadstatic} of the equations of motion are given in the manual \cite{manual}.
    \begin{equation}
        \begin{aligned}
            \ddot x + \left(\frac{q}{mr_0^2}\right)\Phi_0\cdot x= 0 \\
            \ddot y - \left(\frac{q}{mr_0^2}\right)\Phi_0\cdot y= 0 \\
            \ddot z = 0 
            \label{eq:quadstatic}
        \end{aligned}
    \end{equation}
    We see that the ions will move uniformly in the $z$ direction (along the electric field) but will follow a sinusoidal path in the $xz$-plane and an exponential path in the $yz$-plane which will direct them into the walls. To mitigate this problem we will also apply an alternating voltage, this will give us the following equations of motion \eqref{eq:quadalt}:
    \begin{equation}
        \begin{aligned}
            \ddot x + \left(\frac{q}{mr_0^2}\right)(U-V\cos(\omega t))\cdot x= 0 \\
            \ddot y - \left(\frac{q}{mr_0^2}\right)(U-V\cos(\omega t))\cdot y= 0 \\
            \ddot z = 0 
            \label{eq:quadalt}
        \end{aligned}
    \end{equation}
    where $U$ is the DC-offset and $V$ the amplitude of the alternating voltage.
    The $y$-axis will act as a low-pass and the $x$-axis as a high pass, together they form a bandpass filter that only lets certain charge-to-mass ratios pass through to the detector. By adjusting the DC-offset and the alternating voltage amplitude we are able to create a stable path, that lets just one component pass through. The parameters $a$ and $q$ \eqref{eq:params} are defined in the manual \cite{manual}.
    \begin{equation}
        \begin{aligned}
            a = \frac{4eU}{m\omega^2r_0^2} \\
            q = \frac{2eV}{m\omega^2r_0^2}
            \label{eq:params}
        \end{aligned}
    \end{equation}
    The parameter $v=a/q=2U/V$ is independent of the charge-to-mass ratio and is describes the slope in the parameter space as shown in figure \ref{fig:paramspace}.
    \begin{figure}[h!]
    \centering
    \includegraphics[width=0.5\textwidth]{Report/pictures/paramspace.png}
    \caption{The parameter space described by $a$ and $q$. \cite{manual}}
    \label{fig:paramspace}
    \end{figure}
    
    % write something about resolution and delta m
    
    
    \subsection{Detectors}
    \subsubsection{{\scshape Faraday} Detector}
    
    \subsubsection{Secondary Electron Multiplier (SEM)}
    \FloatBarrier
    
    \newpage
\section{Method} \label{sec:method}
\subsection{Setup}
\subsection{Data Set}
\subsection{Curve Fit}
\subsection{Error Estimation}



    \FloatBarrier
    
    \newpage
\section{Execution and Results} \label{sec:results}
    \subsection{Optimization of Settings}
    Before we started with the actual measurements, we optimized the settings.
    First of all, the parameter $\Delta m$, the resolution \texttt{res} and acceleration voltage $U$ had to be chosen optimally. Additionally, the two detector types Faraday and SEM were compared. The residual gas spectrum of the vacuum chamber was recorded for different settings. Since we assume that the residual gas consists primarily of air, we excluded atoms heavier than 50~amu and limited our scans to the range of 1 to 50 amu. For the step sizes we chose 0.2 amu and carried out 7 measurement runs. If nothing else is written, 70 V was selected as $U$.
    
    $\Delta m$ setting affects the offset of the line. To optimize $\Delta m$ we set \texttt{res} = 0\% and performed measurements for $\Delta m = -10\%, -5\%, 0\%, 5\%, 10\%, 15\%, 20\%, 35\%$.  The curves for $\Delta m = -10\%, 5\%, 20\%, 35\%$ are shown in the figures .... for the areas where they differ significantly from 0.
    
    When looking at these curves we notice that the larger $\Delta m$ gets, the less pressure is measured. So the sensitivity decreases strongly with increasing $\Delta m$. Additionally we can see
    that the smaller $\Delta m$ is chosen, the wider the curves become. In order to avoid an overlapping of the individual curves in case of neighboring mass peaks, $\Delta m$ must not be too large. We chose $\Delta m= 20\%$ as the optimal mixture of sensitivity and curve width. 
    
    The smaller the atomic mass, the stronger the sensitivity increases with decreasing $\Delta m$. This is especially visible in the range of amu = 1, where the peak with decreasing $\Delta m$ becomes larger than the peak in the range of amu = 2. This is in accordance with the descriptions of the parameter $\Delta m$ in the manual \cite{manual}. 
    
    To optimize \texttt{res}, which affects the slope of the work line, we chose $\Delta m = 20\%$ and varied \texttt{res} = -5\%, 5\%, 10\%. The results at the relevant ranges are shown in figures. 
    
    We find that \texttt{res} has hardly any influence on the sensitivity close to amu = 0, but a clearly discernible influence in the range of amu = 44. This fact was also described in the manual. Since \texttt{res} does not have a strong influence even in the larger range, we decided to set \texttt{res} in the later measurements neutral to 0\%. 
    
    To select the optimum $U$, $\Delta m = 20 \%$, \texttt{res} = 5\% was kept constant and the voltages of 50~V, 70~V and 90~V were tested. For most of the peaks the sensitivity was higher the lower the $U$ was (see). This may have to do with the fact that at low voltage less double ionization occurs. Therefore we decided to use an $U$ of 50~V.  
    
    Finally we now come to compare the two detectors. For this purpose, both detectors were measured with \texttt{res} = 5\%, $\Delta m = 20\%$ and $U$ = 70~V. As we can see from the figure ..., the sensitivity of the Faraday detector is significantly higher than that of the SEM. Therefore we decided to use the Faraday detector for our measurements. 
    
    In summary, the parameters are optimal if $\Delta m = 20\%$, \texttt{res} = 0\%, $U$ = 50~V and the faraday detector is selected. 
    
    \subsection{Residual gas Identification}
    \subsection{Noble Gas Analysis}
    \begin{figure}[h!]
    \centering
    \includegraphics[width=0.5\textwidth]{Report/pictures/cartridge.JPG}
    \caption{The setup with an attached noble gas cartridge.}
    \label{fig:noblegas}
    \end{figure}
    
    \subsection{Inhaled Air}
    \begin{figure}
        \centering
        \includegraphics[width=\textwidth]{Report/DataResultsPlots/air.pdf}
        \caption{CO$_2$ and O$_2$ amount in Inhaled air.}
        \label{fig:air}
    \end{figure}
    
    
    
    \subsection{Measurement of Ethanol}
    
    \begin{figure}[h!]
    \centering
    \includegraphics[angle=-90, origin=c, width=0.5\textwidth]{Report/pictures/liquids.JPG}
    \caption{The setup to attach a liquid sample to the system. The gaseous phase will be pumped to the mass spectrometer while the liquid phase will stay inside the container.}
    \label{fig:ethanol}
    \end{figure}
    
    \subsection{Deo and Sparkling Water Analysis}
    
    \begin{figure}[h]
            \centering
            \subfloat[balloon with CO2 from sparkling water]{\label{figure:sparkling}
                \includegraphics[width=0.47\textwidth, angle=270, origin=c]{Report/pictures/ballon_bottle2.JPG}}\quad
            \subfloat[balloon attached to the system]{\label{figuer:sparkling2}
                \includegraphics[width=0.47\textwidth]{Report/pictures/ballon_setup.JPG}}\quad
            \caption{The setup of the experiment with samples filled into a balloon.}
            \label{fig:setup2}
    \end{figure}
    \FloatBarrier

    \newpage
    \pagenumbering{gobble} % keine Seitenzahl
    
    \printbibliography %[title={Literaturverzeichnis}]
    
    \appendix
    \addappheadtotoc
    \newpage
    \pagenumbering{roman} % Seitenzahlen i, ii, iii, iv, etc.
    
   % \input{plots.tex}
    %\FloatBarrier
    \section{Error calculation}
\label{app:err}
\subsection{Calibration error}
\label{app:err_cal}
The error $\sigma_{m}$ on the calibration can be derived using error propagation as shown in \eqref{eq:cal_err}.
\begin{align}
    \sigma_m = \sqrt{ \left(\frac{\partial m}{\partial m_\text{Ag, fit}}\cdot \sigma_{m_\text{Ag, fit}}\right)^2 + \left(\frac{\partial m}{\partial m_\text{Xe, fit}}\cdot \sigma_{m_\text{Xe, fit}}\right)^2}
    \label{eq:cal_err}
\end{align}
with
\begin{align*}
     \frac{\partial m}{\partial m_\text{Ag, fit}} &= \frac{m_\text{Xe, true} - m_\text{Ag, true}}{(m_\text{Xe, fit} - m_\text{Ag, fit})^2} \cdot (m-m_\text{Xe, fit}) \\
    \frac{\partial m}{\partial m_\text{Xe, fit}} &= \frac{m_\text{Xe, true} - m_\text{Ag, true}}{(m_\text{Xe, fit} - m_\text{Ag, fit})^2} \cdot (m-m_\text{Ag, fit})
\end{align*}
    \section{Exercises} \label{app:ex}
\subsection{Exercise 8.1}
As a theoretical exercise we had to calculate the partial pressures of a gas mixture with Methane, Nitrogen and Propane. The following peaks have been measured: 
\begin{itemize}
    \item $u_{16} = 1.0012$ V on mass 16
    \item $u_{28} = 0.896$ V on mass 28
    \item $u_{29} = 0.753$ V on mass 29
\end{itemize}
The sensitivities are given as
\begin{itemize}
    \item Methane on mass 16: $\sigma_{16} = 10$ V/Torr
    \item Nitrogen on mass 28: $\sigma_{28} = 15$ V/Torr
    \item Propane on mass 29: $\sigma_{29} = 5$ V/Torr
\end{itemize}
And the following peak ratios are given
\begin{itemize}
    \item Methane: $v_{28/16} = 0$, \qquad \quad\;\; $v_{29/16} = 0$
    \item Nitrogen: $v_{29/28} = 0.00769$,  \quad $v_{16/28} = 0$
    \item Propane: $v_{28/29} = 0.595$,  \;\qquad $v_{16/29} = 0.0016$
\end{itemize}
We calculate:
\begin{align}
    p_\text{Methane} &= (u_{16} - v_{16/29}\cdot u_{29})/\sigma_{16}\\
    p_\text{Nitrogen} &= (u_{28} - v_{28/29}\cdot u_{29} + v_{29/28}\cdot u_{28})/\sigma_{28} \\
    p_\text{Propane} &= (u_{29} - v_{29/28}\cdot u_{28} + v_{16/29}\cdot u_{29} + v_{28/29}\cdot u_{29})/\sigma_{16}
\end{align}
which then yields:
$$ p_\text{Methane} = 0.1 \;\text{Torr}  \qquad  p_\text{Nitrogen} = 0.0305 \;\text{Torr} \qquad  p_\text{Propane} = 0.2397 \;\text{Torr}$$

\pagebreak
\subsection{Exercise 8.2}
For the second exercise we consider a probe with mass $m=7 \,\si{\milli\gram}$ brought back by the Apollo XVII mission. Argon was extracted and the following peak ratios have been determined:
\begin{itemize}
    \item $v_{40/36} = 0.98$
    \item $v_{40/38} = 5.191$
\end{itemize}
As a next step $1.415\cdot 10^{-5}$ cc (STP) ordinary Argon found in air has been added and the new peak ratios have been determined:
\begin{itemize}
    \item $v_{40/36}' = 7.747$
    \item $v_{40/38}' = 40.86$
\end{itemize}
The blank of the system is given as $10^{-8}$ cc (STP) and the discrimination as $D(36/40)=0.998$.
To calculate the amount of $^{40}$Ar in cc/g and the true $^{40}$Ar/$^{36}$Ar and $^{40}$Ar/$^{38}$Ar ratios, we first solved the equation system
\begin{align}
    v_{40/36}\cdot D(36/40) &= \frac{u_{40} + 0.9966\cdot 10^{-8}}{u_{36} + 0.00337\cdot 10^{-8}}\\
    v_{40/36}'\cdot D(36/40) &= \frac{u_{40} + 0.9966\cdot(10^{-8} + 1.415 \cdot 10^{-5})}{u_{36} + 0.00337\cdot(10^{-8} + 1.415 \cdot 10^{-5})}
\end{align}
which gave us the amount of $^{40}$Ar and $^{36}$Ar in the sample:
$$ u_{40} = 1.98\cdot10^{-6} \text{cc} \qquad u_{36} = 2.03\cdot10^{-6} \text{cc} $$
upon solving
\begin{align}
    v_{40/38} = \frac{u_{40} + 0.9966\cdot10^{-8}}{u_{38} + 0.00063\cdot10^{-8}}
\end{align}
we get
$$ u_{38} = 3.83\cdot10^{-7} \text{cc} $$
From that we can calculate the amount of $^{40}$Ar in the sample:
$$u_{40}/m = 0.00028 \;\text{cc/g}$$
and the ratios:
$$^{40}\text{Ar}/^{36}\text{Ar} = 0.97 \qquad ^{40}\text{Ar}/^{38}\text{Ar} = 5.17$$


    % \section{Data}
% \label{app:data}
% The measured data acquired from the LabView program is listed in table \ref{tab:data}.
%     \begin{table}[h!]
%       \begin{center}
%       \DTLsetseparator{,}
%         \DTLloaddb[keys={omega,f,err}]{dat}{table.txt}
%         \begin{tabular}{c|c}
%             \toprule $\omega$ [\si{\deg\per\second}] & $f$ [\si{\hertz}]
%             \DTLforeach{dat}{\o=omega,\f=f,\err=err}
%             {\DTLiffirstrow{\\ \midrule}{\\}
%             \o & \pgfmathprintnumber[textnumber]\f~$\pm$~\pgfmathprintnumber[textnumber]\err }
%             \\\bottomrule
%         \end{tabular}
%         \caption{Data}
%        \label{tab:data}
%       \end{center}
%     \end{table}
    \section{Python Scripts}
\label{app:python}
    The code for the analysis has been written in Python and can be found on \url{https://github.com/hacknus/MassSpectrometer}. The raw and processed datasets as well as the source code of the vector graphics in this LaTeX  report are also uploaded in the same repository.
    
  

    \FloatBarrier
\end{document}